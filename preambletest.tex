%%%%%%%%%%%%%%%%%%%%%%%%%%%%%%%%%
% PACKAGE IMPORTS
%%%%%%%%%%%%%%%%%%%%%%%%%%%%%%%%%
\usepackage{amsmath,amsfonts,amsthm,amssymb,mathtools}
\usepackage{fancyhdr}
\usepackage{enumitem}
\usepackage{bookmark}
\usepackage{xfrac}
\usepackage{graphicx}
\usepackage[english]{babel}
\usepackage{mathrsfs}
\usepackage{centernot}
\usepackage{geometry}
\usepackage{titlesec}
\usepackage{etoolbox}
\usepackage{tocloft}
\usepackage{xcolor}
\usepackage{varwidth}
\usepackage{varwidth}
\usepackage{etoolbox}
\usepackage{multicol,array}
\usepackage{xifthen}
\usepackage{pdfpages}
\usepackage{tcolorbox}
\usepackage{transparent}
\usepackage{ragged2e}
\usepackage{tikzsymbols}
\renewcommand\qedsymbol{$\blacksquare$}
\usepackage{blindtext}
\usepackage{lipsum}
\usepackage[utf8]{vietnam}
\usepackage[T1]{fontenc}
\usepackage{cases}




%%%%%%%%%%%%%%%%%%%%%%%%%%%%%%
% SELF MADE COLORS
%%%%%%%%%%%%%%%%%%%%%%%%%%%%%%
\definecolor{mypbg}{RGB}{215, 216, 217}

%================================
% THEOREM BOX
%================================
% \begin{center}
% 	\parbox{0.8\textwidth}{
% 	\textbf{First Uniquesness Theorem} bla bla bla
% 	}
% \end{center}

%================================
% PROPOSITION
%================================

\NewTColorBox{proposition}{m}{
  standard jigsaw,
  sharp corners,
  boxrule=0.4pt,
  coltitle=black,
  colframe=black,
  opacityback=0,
  opacitybacktitle=0,
  fonttitle=\normalfont\bfseries\sffamily,
  fontupper=\sffamily,
  title={Proposition #1},
  after title={.},
  attach title to upper={\ },
}

%================================
% CLAIM
%================================
% tương tự như Theorem

%================================
% Example Box
%================================
% \tcbuselibrary{theorems,skins,hooks}
% \NewTColorBox{Example}{m}{
%   standard jigsaw,
%   sharp corners,
%   boxrule=0pt,
%   coltitle=black,
%   colframe=black,
%   borderline north = {0.5pt}{0pt}{black},
% 	borderline south = {0.5pt}{0pt}{black},
%   opacityback=0,
%   opacitybacktitle=0,
%   fonttitle=\normalfont\bfseries\upshape,
%   fontupper=\normalfont,
%   title={Example #1},
%   after title={.},
%   attach title to upper={\ },
% }
\tcbuselibrary{theorems,skins,hooks}
\newtcbtheorem[number within=section]{Example}{Example}
{%
	enhanced
	,frame hidden
	,boxrule = 0sp
	,borderline north = {0.5pt}{0pt}{black}
	,borderline south = {0.5pt}{0pt}{black}
  ,opacityback=0
	,sharp corners
	,detach title
	,before upper = \tcbtitle\par\smallskip
	,coltitle = black
	,fonttitle = \bfseries\sffamily
	,description font = \mdseries
	,separator sign none
	% ,segmentation style={solid, myb!85!black}
}
{th}

%================================
% Problem
%================================

\NewTColorBox{problem}{m}{
  standard jigsaw,
  sharp corners,
  boxrule=0pt,
  coltitle=black,
  colback=gray,
  colframe=black,
  opacityback=0.15,
  opacitybacktitle=0,
  fonttitle = \bfseries\sffamily,
  fontupper=\sffamily,
  title={Problem #1},
  after title={.},
  attach title to upper={\ },
}

%%%%%%%%%%%%%%%%%%%%%%%%%%%%%%
% SELF MADE COMMANDS
%%%%%%%%%%%%%%%%%%%%%%%%%%%%%%
\newcommand{\pf}[2]{\begin{proof}[#1]#2\end{proof}}
\newcommand{\ex}[2]{\begin{Example}{#1}{}#2\end{Example}}
\newcommand{\prop}[2]{\begin{proposition}{#1}{}#2\end{proposition}}
\newcommand{\prob}[2]{\begin{problem}{#1}{}#2\end{problem}}
\newcommand{\gachngang}{\noindent\makebox[\linewidth]{\rule{\linewidth}{0.4pt}}}


\DeclarePairedDelimiter{\abs}{\lvert}{\rvert}


%%%%%%%%%%%%%%%%%%%%%%%%%%%%%%
% TOC
%%%%%%%%%%%%%%%%%%%%%%%%%%%%%%
\renewcommand{\cftsecleader}{\cftdotfill{\cftdotsep}}
\renewcommand{\cftdot}{}

\renewcommand{\contentsname}{CONTENTS}
\renewcommand{\cfttoctitlefont}{\normalsize\hfill}
\renewcommand{\cftaftertoctitle}{\hfill}
\setlength{\cftaftertoctitleskip}{0pt}
\setlength{\cftbeforesecskip}{0pt}

% % Set the color for sections and subsections in the table of contents
\renewcommand{\cftsecfont}{\color{blue}}          % Section titles in blue
\renewcommand{\cftsubsecfont}{\color{blue}}        % Subsection titles in red
\renewcommand{\cftsubsubsecfont}{\color{blue}}
% Optional: Set the color of section and subsection page numbers
\renewcommand{\cftsecpagefont}{\color{blue}}
\renewcommand{\cftsubsecpagefont}{\color{black}}
\renewcommand{\cftsubsubsecpagefont}{\color{black}}
% Set the font style for section titles in the TOC
\renewcommand{\cftsecfont}{\bfseries\normalsize\color{blue}} % Bold, large, blue
\renewcommand{\cftsecpagefont}{\normalsize\color{black}} % Same for section page numbers
\renewcommand{\cftsubsecpagefont}{\normalsize\color{black}}


% Page setup

\setlength{\cftsubsecindent}{3em}

\geometry{top=1in, bottom=1in, left=1in, right=1in}

% Define default fancy header style for other pages
\pagestyle{fancy}
\fancyhf{} % Clear all header and footer fields for fancy style
\fancyhead[LE,RO]{\thepage} % Page numbers: Left for even pages, Right for odd pages
\fancyhead[CE]{\textsl{Alexander A. Schekochihin}} % Center header for even pages
\fancyhead[CO]{\textsl{Oxford MMathPhys Lectures: Plasma Kinetics and MHD}} % Center header for odd pages
\renewcommand{\headrulewidth}{0pt} % Disable header line for all pages
\renewcommand{\footrulewidth}{0pt} % Disable footer line for all pages

% Directly define the header and footer for the first page
\fancypagestyle{firstpage}{
    \fancyhf{}
    \fancyhead[L]{\small \textit{DRAFT}} % Left header text
    \fancyhead[R]{\thepage}
    \fancyfoot[C]{E-mail: phamquangchinh07@gmail.com} % Center footer text
    \renewcommand{\footrulewidth}{1pt}
}

\titleformat{\section}
{\large\bfseries}
{\thesection.}{0.5em}{}
\titlespacing*{\section}{0pt}{*0}{*1}

\titleformat{\subsection}
{\normalfont\centering}  % Regular font for the section number
{\thesubsection.}{0.6em}{\slshape}  % Slanted font for the title text only
\titlespacing*{\subsection}{0pt}{*5}{*1}

\titleformat{\subsubsection}
{\normalfont}  % Regular font for the section number
{\thesubsubsection.}{0.6em}{\slshape}  % Slanted font for the title text only
\titlespacing*{\subsubsection}{0pt}{*5}{*1}