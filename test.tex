\documentclass[twoside,a4paper,12pt]{article} % Enable two-sided layout
\usepackage[colorlinks=true, linkcolor=blue]{hyperref}

%%%%%%%%%%%%%%%%%%%%%%%%%%%%%%%%%
% PACKAGE IMPORTS
%%%%%%%%%%%%%%%%%%%%%%%%%%%%%%%%%
\usepackage{amsmath,amsfonts,amsthm,amssymb,mathtools}
\usepackage{fancyhdr}
\usepackage{enumitem}
\usepackage{bookmark}
\usepackage{xfrac}
\usepackage{graphicx}
\usepackage[english]{babel}
\usepackage{mathrsfs}
\usepackage{centernot}
\usepackage{geometry}
\usepackage{titlesec}
\usepackage{etoolbox}
\usepackage{tocloft}
\usepackage{xcolor}
\usepackage{varwidth}
\usepackage{varwidth}
\usepackage{etoolbox}
\usepackage{multicol,array}
\usepackage{xifthen}
\usepackage{pdfpages}
\usepackage{tcolorbox}
\usepackage{transparent}
\usepackage{ragged2e}
\usepackage{tikzsymbols}
\renewcommand\qedsymbol{$\blacksquare$}
\usepackage{blindtext}
\usepackage{lipsum}
% \usepackage[utf8]{vietnam}
\usepackage[T1]{fontenc}
\usepackage{cases}

%%%%%%%%%%%%%%%%%%%%%%%%%%%%%
% TOC
%%%%%%%%%%%%%%%%%%%%%%%%%%%%%
\renewcommand{\cftsecleader}{\cftdotfill{\cftdotsep}}
\renewcommand{\cftdot}{}

\renewcommand{\cfttoctitlefont}{\normalsize\hfill}
\renewcommand{\cftaftertoctitle}{\hfill}
\setlength{\cftaftertoctitleskip}{0pt}
\setlength{\cftbeforesecskip}{0pt}

% % Set the color for sections and subsections in the table of contents
\renewcommand{\cftsecfont}{\color{blue}}          % Section titles in blue
\renewcommand{\cftsubsecfont}{\color{blue}}        % Subsection titles in red
\renewcommand{\cftsubsubsecfont}{\color{blue}}
% Optional: Set the color of section and subsection page numbers
\renewcommand{\cftsecpagefont}{\color{blue}}
\renewcommand{\cftsubsecpagefont}{\color{black}}
\renewcommand{\cftsubsubsecpagefont}{\color{black}}
% Set the font style for section titles in the TOC
\renewcommand{\cftsecfont}{\bfseries\normalsize\color{blue}} % Bold, large, blue
\renewcommand{\cftsecpagefont}{\normalsize\color{black}} % Same for section page numbers
\renewcommand{\cftsubsecpagefont}{\normalsize\color{black}}

%%%%%%%%%%%%%%%%%%%%%%%%%%%%%%
% SELF MADE COLORS
%%%%%%%%%%%%%%%%%%%%%%%%%%%%%%
\definecolor{mypbg}{RGB}{215, 216, 217}

%================================
% THEOREM BOX
%================================
% \begin{center}
% 	\parbox{0.8\textwidth}{
% 	\textbf{First Uniquesness Theorem} bla bla bla
% 	}
% \end{center}

%================================
% PROPOSITION
%================================

\NewTColorBox{proposition}{m}{
  standard jigsaw,
  sharp corners,
  boxrule=0.4pt,
  coltitle=black,
  colframe=black,
  opacityback=0,
  opacitybacktitle=0,
  fonttitle=\normalfont\bfseries\sffamily,
  fontupper=\sffamily,
  title={Proposition #1},
  after title={.},
  attach title to upper={\ },
}

%================================
% CLAIM
%================================
% tương tự như Theorem

%================================
% Example Box
%================================
% \tcbuselibrary{theorems,skins,hooks}
% \NewTColorBox{Example}{m}{
%   standard jigsaw,
%   sharp corners,
%   boxrule=0pt,
%   coltitle=black,
%   colframe=black,
%   borderline north = {0.5pt}{0pt}{black},
% 	borderline south = {0.5pt}{0pt}{black},
%   opacityback=0,
%   opacitybacktitle=0,
%   fonttitle=\normalfont\bfseries\upshape,
%   fontupper=\normalfont,
%   title={Example #1},
%   after title={.},
%   attach title to upper={\ },
% }
\tcbuselibrary{theorems,skins,hooks}
\newtcbtheorem[number within=section]{Example}{Example}
{%
	enhanced
	,frame hidden
	,boxrule = 0sp
	,borderline north = {0.5pt}{0pt}{black}
	,borderline south = {0.5pt}{0pt}{black}
  ,opacityback=0
	,sharp corners
	,detach title
	,before upper = \tcbtitle\par\smallskip
	,coltitle = black
	,fonttitle = \bfseries\sffamily
	,description font = \mdseries
	,separator sign none
	% ,segmentation style={solid, myb!85!black}
}
{th}

%================================
% Problem
%================================

\NewTColorBox{problem}{m}{
  standard jigsaw,
  sharp corners,
  boxrule=0pt,
  coltitle=black,
  colback=gray,
  colframe=black,
  opacityback=0.15,
  opacitybacktitle=0,
  fonttitle = \bfseries\sffamily,
  fontupper=\sffamily,
  title={Problem #1},
  after title={.},
  attach title to upper={\ },
}

%%%%%%%%%%%%%%%%%%%%%%%%%%%%%%
% SELF MADE COMMANDS
%%%%%%%%%%%%%%%%%%%%%%%%%%%%%%
\newcommand{\pf}[2]{\begin{proof}[#1]#2\end{proof}}
\newcommand{\ex}[2]{\begin{Example}{#1}{}#2\end{Example}}
\newcommand{\prop}[2]{\begin{proposition}{#1}{}#2\end{proposition}}
\newcommand{\prob}[2]{\begin{problem}{#1}{}#2\end{problem}}
\newcommand{\gachngang}{\noindent\makebox[\linewidth]{\rule{\linewidth}{0.4pt}}}


\DeclarePairedDelimiter{\abs}{\lvert}{\rvert}





% Page setup

\setlength{\cftsubsecindent}{3em}

\geometry{top=1in, bottom=1in, left=1in, right=1in}

% Define default fancy header style for other pages
\pagestyle{fancy}
\fancyhf{} % Clear all header and footer fields for fancy style
\fancyhead[LE,RO]{\thepage} % Page numbers: Left for even pages, Right for odd pages
\fancyhead[CE]{\textsl{Alexander A. Schekochihin}} % Center header for even pages
\fancyhead[CO]{\textsl{Oxford MMathPhys Lectures: Plasma Kinetics and MHD}} % Center header for odd pages
\renewcommand{\headrulewidth}{0pt} % Disable header line for all pages
\renewcommand{\footrulewidth}{0pt} % Disable footer line for all pages

% Directly define the header and footer for the first page
\fancypagestyle{firstpage}{
    \fancyhf{}
    \fancyhead[L]{\small \textit{DRAFT}} % Left header text
    \fancyhead[R]{\thepage}
    \fancyfoot[C]{E-mail: phamquangchinh07@gmail.com} % Center footer text
    \renewcommand{\footrulewidth}{1pt}
}

\titleformat{\section}
{\large\bfseries}
{\thesection.}{0.5em}{}
\titlespacing*{\section}{0pt}{*0}{*1}

\titleformat{\subsection}
{\normalfont\centering}  % Regular font for the section number
{\thesubsection.}{0.6em}{\slshape}  % Slanted font for the title text only
\titlespacing*{\subsection}{0pt}{*5}{*1}

\titleformat{\subsubsection}
{\normalfont}  % Regular font for the section number
{\thesubsubsection.}{0.6em}{\slshape}  % Slanted font for the title text only
\titlespacing*{\subsubsection}{0pt}{*5}{*1}

\begin{document}
% \renewcommand{\cftsecleader}{\cftdotfill{\cftdotsep}}
% \renewcommand{\cftdot}{}

\renewcommand{\contentsname}{CONTENTS}
% \renewcommand{\cfttoctitlefont}{\normalsize\hfill}
% \renewcommand{\cftaftertoctitle}{\hfill}
% \setlength{\cftaftertoctitleskip}{0pt}
% \setlength{\cftbeforesecskip}{0pt}

% % % Set the color for sections and subsections in the table of contents
% \renewcommand{\cftsecfont}{\color{blue}}          % Section titles in blue
% \renewcommand{\cftsubsecfont}{\color{blue}}        % Subsection titles in red
% \renewcommand{\cftsubsubsecfont}{\color{blue}}
% % Optional: Set the color of section and subsection page numbers
% \renewcommand{\cftsecpagefont}{\color{blue}}
% \renewcommand{\cftsubsecpagefont}{\color{black}}
% \renewcommand{\cftsubsubsecpagefont}{\color{black}}
% % Set the font style for section titles in the TOC
% \renewcommand{\cftsecfont}{\bfseries\normalsize\color{blue}} % Bold, large, blue
% \renewcommand{\cftsecpagefont}{\normalsize\color{black}} % Same for section page numbers
% \renewcommand{\cftsubsecpagefont}{\normalsize\color{black}}

% Title Section
\begin{center}
    \thispagestyle{firstpage} % Apply firstpage style to the entire first page
    \vspace*{10pt} % Adjust this for top spacing
    {\LARGE \textbf{Lectures on Kinetic Theory and\\ Magnetohydrodynamics of Plasmas}} \\[1em]
    {\small{\textbf{(Oxford MMathPhys/MSc in Mathematical and Theoretical Physics)}}} \\[1.5em]
    {\large \textbf{ Alexander A. Schekochihin}} \\[0.5em]
    {\small The Rudolf Peierls Centre for Theoretical Physics, University of Oxford, Oxford OX1 3NP, UK}\\
    {\small Merton College, Oxford OX1 4JD, UK}\\[1em]
    {\small \textit{(compiled on 5 October 2024)}}
\end{center}

\noindent These are the notes for my lectures on Ordinary Differential Equations for 1st-year
undergraduate physicists, taught in 2018-22 as part of Paper CP3 at Oxford. They also
include lectures on Normal Modes (part of Paper CP4), taught since 2021. I will be
grateful for any feedback from students, tutors, or (critical) sympathizers. The latest,
up-to-date version of this file is available at\\

\noindent\makebox[\linewidth]{\rule{\linewidth}{0.4pt}}
\begin{center}
\parbox{1\textwidth}{
\emph{Below I give references to some relevant bits of various books and online lecture notes. Having access to these materials is not essential for being able to follow these lectures, but reading (these or other) books above and beyond your lecturer’s notes is essential for you to be able to claim that you have received an education.}
}
\end{center}
\noindent\makebox[\linewidth]{\rule{\linewidth}{0.4pt}}
\tableofcontents

\newpage
\begin{flushright}
    \blindtext[1]
\end{flushright}

\section{The Language of the Game}



\subsection{What is an ODE?}



\subsection{Integral Curves and the Cauchy problems}
\subsubsection{First Subsubsec}


\subsection{Existence and uniqueness}

\section{Another Section}
\lipsum[1]
\begin{center}
	\parbox{0.8\textwidth}{
	\textbf{First Uniquesness Theorem}. a fat cat just a fat cat to be a fat cat just eat like a fat cat
	}
\end{center}
\section{Yet Another Section}
\prob{12.2}{A function fat kill a cat to be fat}
\pf{Proof}{It begins with a knife to kill a cat}
\prop{12.2.4}{Shitto a schizo frenzy}
\section{Final Section}
\ex{HEllo con dog}{Lorem ipsum dolor sit amet, consectetuer adipiscing elit. Ut purus elit, vestibulum ut,
placerat ac, adipiscing vitae, felis}

\newpage

\lipsum[1]
\end{document}
